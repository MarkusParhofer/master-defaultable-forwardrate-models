\documentclass[12pt]{article}
\usepackage[a4paper, width=6.3in, left=1.3in, right=1.3in, height=9.2in, top=1.1in]{geometry}
\usepackage[T1]{fontenc}
\usepackage[utf8]{inputenc}
\usepackage{lmodern}
\usepackage{amsfonts,amsmath,amssymb,amsthm}
\usepackage[htt]{hyphenat}
\usepackage{fancyhdr}
\usepackage[nottoc,notlot,notlof]{tocbibind}
\usepackage{verbatim}
\usepackage{graphicx}
\usepackage{wrapfig}
\usepackage{caption}
\usepackage[table]{xcolor}
\fancyfoot[R]{\thepage}
\usepackage{ctable}
\usepackage{color} %red, green, blue, yellow, cyan, magenta, black, white
\usepackage[title, titletoc]{appendix}
\usepackage{array}
\usepackage{hyperref}
\usepackage{listings}
\usepackage[noabbrev]{cleveref}
\usepackage{subfig}

\renewcommand{\thesubfigure}{Figure \arabic{subfigure}}
\captionsetup[subfigure]{labelformat=simple, labelsep=colon}

\newcolumntype{C}[1]{>{\centering\arraybackslash}p{#1}}

\definecolor{mygreen}{RGB}{28,172,0} % color values Red, Green, Blue
\definecolor{mydarkgreen}{RGB}{28,132,0} 
\definecolor{mylilas}{RGB}{170,55,241}
\definecolor{mygrey}{RGB}{123,123,123}
\definecolor{myorange}{RGB}{230,80,80}
\definecolor{codebackground}{RGB}{230,230,230}
\definecolor{mydarkblue}{RGB}{0,102,153}


\newcommand{\mbeq}{\overset{!}{=}}

% Paths to java files. Note these might give errors when working on a different computer
\newcommand{\finmathlib}{C:/Users/marku/IdeaProjects/finmath-lib/src/main/java/net/finmath}
\newcommand{\libMC}{\finmathlib/montecarlo}
\newcommand{\codepath}{C:/Users/marku/IdeaProjects/master-thesis-default-forward-rate-models/src}
\newcommand{\codepathmodels}{\codepath/main/java/info/quantlab/masterthesis}
\newcommand{\codedlmm}[1]{\codepathmodels/defaultablelibormodels/#1}
\newcommand{\codecovdlmm}[1]{\codepathmodels/defaultablecovariancemodels/#1}
\newcommand{\codeprocess}[1]{\codepathmodels/process/#1}

\newcommand{\fig}[1]{figures/#1}

\newcommand{\einsch}{\,\rule[-5pt]{0.4pt}{15pt}\,{}}
%\newcommand{\EQB}[1]{\mathbb{E}^{\mathbb{Q}^B}\left[#1\right]}
\parindent 0ex
\renewcommand{\baselinestretch}{1.5}
\newtheorem{assumption}{Assumption}[section]
\newtheorem{theorem}{Theorem}[section]
\newtheorem{lemma}[theorem]{Lemma}
\newtheorem{notation}[theorem]{Notation}
\newtheorem{remark}[theorem]{Remark}
\newtheorem{definition}[theorem]{Definition}

\crefname{assumption}{assumption}{assumptions}
\crefname{notation}{notation}{notations}
\crefname{remark}{remark}{remarks}


\renewcommand{\labelenumi}{\em{(\roman{enumi})}}
\renewcommand{\labelenumii}{\em{(\alph{enumii})}}
\captionsetup[figure]{skip=0pt}


\lstset{
	inputencoding		=utf8,
	postbreak			=\raisebox{0ex}[0ex][0ex]{\ensuremath{\color{red}\hookrightarrow\space}},
	language			=Java,
	basicstyle			=\ttfamily\scriptsize,
	backgroundcolor		=\color{codebackground},
	tabsize				=3,
	captionpos			=b,
	morekeywords		={java2tikz},
	keywordstyle		=\color{orange},
	morekeywords		=[2]{1},
	keywordstyle		=[2]{\color{black}},
	identifierstyle		=\color{black},
	stringstyle			=\color{mygreen},
	commentstyle		=\color{mydarkgreen},
	showstringspaces	=false,
	showspaces			=false,
	numbers				=left,
	numberstyle			=\tiny\color{black},
	numbersep			=5pt,
	breaklines,
	breakatwhitespace,
	emph=[1]{RandomVariable,ProcessModel,MonteCarloProcess,LIBORMarketModel,DefaultableLIBORMarketModel,SimulationModel,CalculationException},emphstyle=[1]\color{mydarkblue}, %some words to emphasise
	emph=[2]{SPREADS}, emphstyle=[2]\color{mylilas},    
}


\begin{document}
Let $L_t := \left(L_0(t), ..., L_{N-1}(t)\right)$ and $S_t := \left(S_0(t), ..., S_{N-1}(t)\right)$.\\
We show the existence for the SDEs of the log spread, meaning we will show the three conditions for the SDE of $s_t=\log\left(S_t\right)$:
\begin{align*}
	ds^i_t = \left(\tilde{\mu}^i\left(t,L_t, S_t\right) - \frac{\lVert\tilde{\lambda}^i\left(t,L_t,S_t\right)\rVert ^2}{2}\right) dt
	 + \tilde{\lambda}^i\left(t,L_t, S_t\right) \cdot dU_t
\end{align*}
We assume, that the SDEs for $L$ are specified, such that the Lipschitz-continuity and the sub-linear growth conditions are fulfilled and such that they prevent $L_i(t) < c_i$, where $c_i$ is a constant with $c_i\Delta T_i > -1$ for all $i \in \{0, ..., N-1\}$. We also assume that the factor loadings of $\lambda^i(t, L_t)$ depend only on the LIBOR they correspond to: $\lambda^i(t,L_i(t)) := \lambda^i(t,L_t)$
	(e.g.\ a logmormal statespace with bounded volatility is sufficient). \\
We define the box $A = \prod_{i=0}^{N-1}\left[c_i, \infty\right[$. According to our assumptions, it holds that $L_t\in A$ for all $t\in \left[0,\tilde{T}\right]$.\\
Furthermore, we assume that the free parameters are given by $\mathbb{R}$-valued functions $f^{i k}(t,x,s)$ on $\left[0,\tilde{T}\right]\times A \times \mathbb{R}_{>0}^N$ with $f^{i,k}(t,L_t, S_t) := f^{i,k}_t$ and we assume they are Lipschitz in the second and third parameter, and bounded by $K_1 > 0$, i.e.\ $|f^{ik}(t,x,s)| \le K_1$), which yields sub-linear growth.\\

We first prove the following statement for $ x \in A$: there exist a constant $K_2 > 0$ such that
\begin{align*}
	0 < \frac{1}{1+ x_i\Delta T_i } \le \frac{1+|x_i|}{1+ x_i\Delta T_i } \le K_2 \tag{A}
\end{align*}
The first two inequalities are trivial, as $1 + x_i\Delta T_i > 0$.
For the second equation first note that for $x_i \ge 0$ this is always true, because (assuming $x_i \ge 0$):
\begin{align*}
	\frac{1+|x_i|}{1+ x_i\Delta T_i } &= \underbrace{\frac{1}{1+ x_i\Delta T_i }}_{\le 1}  \underbrace{\frac{x_i}{\frac{1}{\Delta T_i}+ x_i }}_{\le 1}\frac{1}{\Delta T_i}\\
	&\le 1 + \frac{1}{\Delta T_i}
\end{align*}
Also note 
\begin{align*}
	\lim\limits_{x_i \rightarrow -\frac{1}{\Delta T_i}} \frac{\overbrace{1 + |x_i|}^{ \ge 1}}{1+ \underbrace{x_i\Delta T_i}_{\rightarrow -1}} = \infty
\end{align*}
Since $\frac{1+|z|}{1+ z\Delta T_i }$ is continuous for all $z > -\frac{1}{\Delta T_i}$, we can (for any $c_i$) find an $\epsilon$, such that $-1 < \epsilon\Delta T_i < c_i \Delta T_i$ with:
\begin{align*}
	K_2:= \frac{1+|\epsilon|}{1+ \epsilon\Delta T_i } > \frac{1+|x_i|}{1+ x_i\Delta T_i } \quad \text{for all } x \in A
\end{align*}
This concludes the proof of statement (A).\\
Let us now prove that for some constant $K_3 > 0$:
\begin{align*}
	\left| y_i \lambda^{ik}(t, x) - x_i \lambda^{ik}(t,y) \right|\le K_3 \;\left| x_i - y_i \right| \left(1 + \left|x_i\right|\right)\left(1 + \left|y_i\right|\right) \tag{B}
\end{align*}
W.l.o.g. we assume $|y_i| < |x_i|$ and get:
\begin{align*}
	\left| y_i \lambda^{ik}(t, x) - x_i \lambda^{ik}(t,y) \right| &= \left| y_i \lambda^{ik}(t, x) - y_i \lambda^{ik}(t,y) + y_i \lambda^{ik}(t,y) - x_i \lambda^{ik}(t,y) \right|\\
	&\le \left| y_i \right| \left|\lambda^{ik}(t, x) - \lambda^{ik}(t,y) \right|+ \left|\lambda^{ik}(t,y)\right| \left|y_i - x_i\right|\\
	& \le C_1 \left| y_i \right| \left|x_i - y_i\right|+ C_2\left(1 + \left|y_i\right|\right) \left|y_i - x_i\right|\\
	&\le \max\left(C_1, C_2\right) \left(1 + \left|x_i\right| + \left| y_i \right| \right) \left|x_i - y_i\right|\\
	&\le \max\left(C_1, C_2\right) \left(1 + \left|x_i\right|\right) \left( 1 +\left| y_i \right| \right) \left|x_i - y_i\right|
\end{align*}
where $C_1$ and $C_2$ are the Lipschitz- resp. sub-linear growth constants of $\lambda^{i k}$. Setting $K_3:=\max\left(C_1, C_2\right)$ yields  the result.

We now prove the conditions for the factor loadings of the spread. Note that by \cref{...} we have:
\begin{align*}
	\tilde{\lambda}^{i k}\left(t,x, s\right) = 
	\left\{
	\begin{aligned}
		&\frac{\lambda^{i k}(t,x) \Delta T_i}{1 + x_i\Delta T_i} \quad &\text{for } k \in \{1,...,m\}\\
		&f^{i k}(t,x,s) \quad &\text{for } k \in \{m+1,...,m^d\}
	\end{aligned}
	\right.
\end{align*}
We can prove the conditions for each component and factor, because they are stable under addition. Note that for $k\in\{m+1, ..., m^d\}$ they are immediately given.\\
We start with sub-linear growth for $k\in\{1,...,m\}$ for all $x \in A$, and for all $s\in \mathbb{R}_{>0}^N$:
\begin{align*}
	\left|\tilde{\lambda}^{i k}\left(t,x, s\right)\right| = \left|\frac{\lambda^{i k}(t,x) \Delta T_i}{1 + x_i\Delta T_i}\right|\le C_2 \frac{1+|x_i|}{1+x_i\Delta T_i}\le C_2\;K_2=:K_4\tag{C}
\end{align*}
where we used the sub-linear growth of $\lambda^{ik}(t,x)$ in the first inequality and (A) in the second. Hence we even have that $\lambda^{ik}$ is bounded for all $i\in\{0,...N-1\}$ and $k\in\{1,...m^d\}$.\\
For the Lipschitz condition we have:
\begin{align*}
	&\left|\frac{\lambda^{i k}(t,x) \Delta T_i}{1 + x_i\Delta T_i} - \frac{\lambda^{i k}(t,y) \Delta T_i}{1 + y_i\Delta T_i}\right| =
	\Delta T_i \left|\dfrac{\lambda^{i k}(t,x)\left(1 + y_i\Delta T_i\right) - \lambda^{i k}(t,y)\left(1 + x_i\Delta T_i\right)}{\left(1 + x_i\Delta T_i\right)\left(1 + y_i\Delta T_i\right)}\right|\\
	&\le \Delta T_i \left|\dfrac{\left|\lambda^{i k}(t,x) - \lambda^{i k}(t,y)\right| + \Delta T_i \left|\lambda^{i k}(t,x) y_i - \lambda^{i k}(t,y)x_i\right|}{\left(1 + x_i\Delta T_i\right)\left(1 + y_i\Delta T_i\right)}\right|\tag{D}\\
	&\le \Delta T_i \;\left|\dfrac{C_1\left|x_i - y_i\right| + \Delta T_i K_3 \left(1 + \left|x_i\right|\right)\left( 1 +\left| y_i \right| \right) \left|x_i - y_i\right|}{\left(1 + x_i\Delta T_i\right)\left(1 + y_i\Delta T_i\right)}\right|\tag{E}\\
	&\le \Delta T_i \;\left|x_i - y_i\right|\left|\dfrac{C_1 + \Delta T_i K_3 \left(1 + \left|x_i\right|\right) \left( 1 +\left| y_i \right| \right)}{\left(1 + x_i\Delta T_i\right)\left(1 + y_i\Delta T_i\right)}\right|\tag{F}\\
	&\le \left|x_i - y_i\right| \underbrace{\Delta T_i \left(C_1 \left(K_2\right)^2 +\Delta T_i  K_3 \left(K_2\right)^2\right)}_{=:K_5},\tag{G}
\end{align*}
where for (D) we used the triangle inequality, (E) comes from (B) and the Lipschitz condition of $\lambda^{i k}$ and (G) from (A). This yields the Lipschitz condition for the factor loadings $\tilde{\lambda}^{ik}$.\\
Next we will prove the conditions also for the drift $\tilde{\mu}^i$ but first we need to pave the way. \\
First note that any multiplication of two factor loadings is again Lipschitz and bounded, i.e. there exist a $K_6 > 0$ and $K_7>0$ such that for all $i, j \in \{0,...,N-1\}$ and $k\in\{1,...,m^d\}$:
\begin{align*}
	\left|\tilde{\lambda}^{i k}\left(t,x, s\right)\tilde{\lambda}^{j k}\left(t,x, s\right) - \tilde{\lambda}^{i k}\left(t,y, u\right)\tilde{\lambda}^{j k}\left(t,y, u\right)\right| &\le K_6\; \lVert(x,s) - (y,u)\rVert \\
	\left|\tilde{\lambda}^{i k}\left(t,x, s\right)\tilde{\lambda}^{j k}\left(t,x, s\right)\right| &\le K_7\tag{H}
\end{align*}
This stems from \cref{...}, together with the fact that both $\tilde{\lambda}^{i k}$ and $\tilde{\lambda}^{j k}$ are bounded.\\
Now we can prove that $\tilde{\mu}^i$ is Lipschitz and has sub linear growth.\\
By \cref{...} we have:
\begin{align*}
	\tilde{\mu}^i_t = &\; \sum_{j=m(t)+1}^{i}\frac{\sum_{k=1}^{m}\lambda^{i k}\left(t, L_t\right) \Delta T_i \lambda^{j k}\left(t, L_t\right) \Delta T_j}{(1 + L_i(t)\Delta T_i) (1 + L_j(t)\Delta T_j)}\\
	&+ \sum_{j=m(t)+1}^{i}\sum_{k=m+1}^{m^d}f^{i k}(t,L_t,S_t)\dfrac{S_j(t)f^{j k}(t,L_t,S_t)}{1 + L^d_j(t)\Delta T_j}
\end{align*}
hence with the relation $L^d_t = L_t + S_t$  we have:
\begin{align*}
	\tilde{\mu}^i\left(t, x, s\right) = &\; \sum_{j=m(t)+1}^{i}\sum_{k=1}^{m}\frac{\lambda^{i k}\left(t, x\right) \Delta T_i \lambda^{j k}\left(t, x\right) \Delta T_j}{(1 + x_i\Delta T_i) (1 + x_j\Delta T_j)}\\
	&+ \sum_{j=m(t)+1}^{i}  \sum_{k=m+1}^{m^d}  f^{i k}(t,x,s)f^{j k}(t,x,s)\dfrac{s_j}{1 + x_j\Delta T_j + s_j\Delta T_j}\\
	=&\; \sum_{j=m(t)+1}^{i}\sum_{k=1}^{m}\tilde{\lambda}^{i k}\left(t, x, s\right)\tilde{\lambda}{j k}\left(t, x, s\right)\tag{I}\\
	&+ \sum_{j=m(t)+1}^{i}  \sum_{k=m+1}^{m^d}  \tilde{\lambda}^{i k}\left(t, x, s\right)\tilde{\lambda}^{j k}\left(t, x, s\right)\underbrace{\frac{s_j}{1 + x_j\Delta T_j + s_j\Delta T_j}}_{=(K)}\tag{J}
\end{align*}
As (I) is just the sum of Lipschitz-continuous and bounded functions, it is again Lipschitz and bonded. Because $x_j\Delta T_j > c_j > -1$ and $s_j \ge 0$, (K) is naturally bounded by some constant $C_3$ with the same arguments as for (A). Also with an exactly analogous proof to (A) we have that $\frac{1 + \left|y_j\right| + \left|v_j\right|}{1 + \Delta T_j(x_j + s_j)}$ is bounded by some constant $C_4$. For Lipschitz continuity we have therefore:
\begin{align*}
	&\left|\frac{s_j}{1 + x_j\Delta T_j + s_j\Delta T_j} - \frac{v_j}{1 + y_j\Delta T_j + v_j\Delta T_j} \right|\\
	&=\left|\frac{(s_j - v_j) + \Delta T_j(s_j y_j - v_jx_j)}{\left(1 + \Delta T_j(x_j + s_j)\right)\left(1 + \Delta T_j(y_j + v_j)\right)}\right|\\
	&\le \left|\frac{\lVert (x_j,s_j) - (y_j, v_j)\rVert + \Delta T_j(\left|y_j\right| \left|s_j - v_j\right| + \left|v_j\right| \left|y_j - x_j\right|)}{\left(1 + \Delta T_j(x_j + s_j)\right)\left(1 + \Delta T_j(y_j + v_j)\right)}\right|\\
	&\le \lVert (x_j,s_j) - (y_j, v_j)\rVert\left|\frac{1 + \Delta T_j(\left|y_j\right| + \left|v_j\right|)}{\left(1 + \Delta T_j(x_j + s_j)\right)\left(1 + \Delta T_j(y_j + v_j)\right)}\right|\\
	&\le \lVert (x_j,s_j) - (y_j, v_j)\rVert\left|\frac{1 + \Delta T_j\left(1 + \left|y_j\right| + \left|v_j\right|\right)\left(1 + \left|x_j\right| + \left|s_j\right|\right)}{\left(1 + \Delta T_j(x_j + s_j)\right)\left(1 + \Delta T_j(y_j + v_j)\right)}\right|\\
	&\le \lVert (x_j,s_j) - (y_j, v_j)\rVert  \left(C_3^2 + C_4^2\right)
\end{align*}
Hence we have that (J) is nothing else than a sum of multiplications of bounded, implying that the drift $\tilde{\mu}^{i}$ is Lipschitz and bounded for all $i\in\{0,...N-1\}$.\\
This concludes the proof, because by (H) we already have $\lVert\tilde{\lambda}^{i}\rVert ^2$ is Lipschitz and bounded, proving
\begin{align*}
	\tilde{\mu}^i\left(t, x, s\right) - \dfrac{\lVert\tilde{\lambda}^{i}(t,x,s)\rVert^2}{2}
\end{align*}
is Lipschitz and bounded.
\end{document}