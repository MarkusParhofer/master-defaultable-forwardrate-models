\documentclass[12pt]{article}
\usepackage[a4paper, width=6.3in, left=1.3in, right=1.3in, height=9.2in, top=1.1in]{geometry}
\usepackage{amsfonts,amsmath,amssymb,amsthm}
\usepackage[none]{hyphenat}
\usepackage{fancyhdr}
\usepackage[nottoc,notlot,notlof]{tocbibind}
\usepackage{verbatim}
\usepackage{graphicx}
\usepackage{wrapfig}
\usepackage{caption}
\usepackage[table]{xcolor}
\fancyfoot[R]{\thepage}
\usepackage{listings}
\usepackage{ctable}
\usepackage{color} %red, green, blue, yellow, cyan, magenta, black, white
\usepackage[title, titletoc]{appendix}
\usepackage{array}

\newcolumntype{C}[1]{>{\centering\arraybackslash}p{#1}}

\definecolor{mygreen}{RGB}{28,172,0} % color values Red, Green, Blue
\definecolor{mylilas}{RGB}{170,55,241}
\definecolor{mygrey}{RGB}{123,123,123}


\newcommand{\einsch}{\,\rule[-5pt]{0.4pt}{15pt}\,{}}
\parindent 0ex
\renewcommand{\baselinestretch}{1.5}
\newtheorem{theorem}{Theorem}[section]
\newtheorem{lemma}[theorem]{Lemma}
\newtheorem{notation}[theorem]{Notation}
\newtheorem{remark}[theorem]{Remark}
\newtheorem{definition}[theorem]{Definition}
\renewcommand{\labelenumi}{\em{(\roman{enumi})}}
\renewcommand{\labelenumii}{\em{(\alph{enumii})}}
\captionsetup[figure]{skip=0pt}


\begin{document}
	\begin{titlepage}
		
		
		
		\begin{center}
			\normalsize{Ludwig-Maximilians-Universität München \\
				Mathematisches Institut}\\[15ex]
			
			\LARGE{Master's Thesis}\\[1.2ex]	
			\LARGE{\textbf{Default Forward Rate Models for the Valuation of Loans including Behavioral Aspects}} \\[2.5ex]
			\Large{\textit{Markus Parhofer}}\\[9ex]
		\end{center}
		
		
		\begin{center}
			\includegraphics[width=1.8in]{siegel}\\[13ex]
		\end{center}
		
		\begin{center}
			\normalsize{}
			Under supervision of\\
			Prof. Dr. Christian Fries\\
			\today\
			\\
		\end{center}
		
	
	
	
\end{titlepage}

\tableofcontents
\thispagestyle{empty}
\clearpage


	
	
	% ------------------------------Content------------------------
	
	
	% ----------- Intro -----------
	
	
	
	\section{Introduction}
	\subsection{Motivation}
	\subsection{Aim of the thesis}
	
	
	
	
	% ------ Defaultable LMM ------
	
	
	
	\pagebreak
	\section{Defaultable LIBOR Market Models}
	In this section we will introduce defaultable LIBOR market models that we can use to value credits and credit options.\\
	Our main source for this section is the article "Defaultable Discrete Forward Rate Model with Covariance Structure guaranteeing Positive Credit Spreads" authored by Christian Fries \cite{FriesDLMM}.\\

	\subsection{The Defaultable Forward Rate}
		We remain in the same setting as in the non-defaultable model, where we have a LIBOR tenor discretization \((T_i)_{i\in\{0, 1, ..., n\}}\) and a set of (non-defaultable) zero coupon bonds \((P(t;T_i))_{i\in\{0, 1, ..., n\}}\). Hence we can define the same products and apply the same valuation formulas.\\
		We extend the model by defining an additional set of zero coupon bonds which are defaultable.
		\((P^d(t;T_i))_{i\in\{0, ..., n\}}\).\\
		Note here, that by construction we must still use the riskless bonds for the calculation of the Numeraire, as one can not use a risky asset as numeraire.
		Furthermore we will - for now - not consider recovery rates, i.e. we assume that a party is either able to pay all or nothing.
		We will now introduce the concept of default and defaultable zero coupon bonds.
	\begin{definition}
		The \emph{default time} is a stopping time \(\tau(\omega)\) on the Filtration \((\mathcal{F}_t)_{t\in \mathbb{R}^+}\).\\
		The \emph{default indicator} \(J(t)\) is the indicator process over the default time:
		\[J(t) := \mathbf{1}_{\{\tau < t\}}\]
	\end{definition}
	\begin{definition}
		% TODO: Stimmt das wirklich? ein Default Process ist eher ein sich anbahnendes Ereignis?
		The \emph{Defaultable Zero Coupon Bond} with price process \[P^d(t; T_i)\] at time \(t \in \left[0, T\right]\) is a traded asset that pays \(1 - J(T_i)\) at maturity  \(T_i \in \{T_0, ..., T_n\}\).\\
		Hence it pays 1 if the default has not happened until maturity. 
	\end{definition}
	It is easy to see, that if default occurs, the price of a defaultable zero coupon bond jumps to zero. This means that the price process can be discontinuous at default events. This gives notion to the definition of a zero coupon bond conditional on pre-default.
	\begin{definition}
		The \emph{Defaultable Zero Coupon Bond conditional pre-default} is a continuous Itô-stochastic process \(P^{d,*}(t; T)\) at time \(t \in \left[0, T\right]\) with maturity \(T \in \mathbb{R}\) such that 
		\[P^{d}(\omega, t; T_i) = P^{d,*}(\omega, t; T_i) \quad \forall\omega \in A_t,\]
		where \(A_t := \{\omega \in \Omega \; | \; \tau(\omega) > t\}\in \mathcal{F}_t\) is the set of all states where default has not happened at time \(t\) and \(T_i \in \{T_0, ... T_n\}\).
	\end{definition}
	\begin{definition}
		The \emph{simple Defaultable Forward Rate} is the rate gained from \(P^{d,*}(t; T)\) by the same concept as in a non-defaultable model:
		\[L^{d}_i(t) := L^{d}(t; T_i, T_{i+1}) = \left( \frac{P^{d,*}(t; T_i)}{P^{d,*}(t; T_{i+1})} - 1\right) \left( T_{i+1} - T_i\right),\]
		where \(T_i \in \{T_0, ... T_n\}\).
	\end{definition}
	The simple Defaultable Forward Rate is the rate at which one can lend money to a defaultable party (for the time period \(T_i\) to \(T_{i+1}\)) at the risk of default, if the defaultable party is not in default at the evaluation time \(t\) \color{red}[Add source]\color{black} % TODO: add a source for this statement.
	.
	\subsection{Covariance Structures Guaranteeing positive spread}
	
	
	
	
	% -------- Pricing of credits with optionalities
	
	
	\pagebreak
	\section{Credit and Credit Option Pricing}
	\subsection{General Credit Pricing}
	\subsection{Pricing Credit Options}
	
	
	
	% -------- Behavioral Aspects
	
	
	\pagebreak
	\section{Introducing Behavioral Aspects}
	\subsection{General Credit Pricing}
	\subsection{Pricing Credit Options}
	
	\pagebreak
	\begin{thebibliography}{}
		
		\bibitem{FriesDLMM}
		Fries, Christian P..
		February 22, 2022. 
		\textit{Defaultable Discrete Forward Rate Model with Covariance Structure guaranteeing Positive Credit Spreads} 
		Available at SSRN: \texttt{<https://ssrn.com/abstract=3667878>} or \texttt{<http://dx.doi.org/10.2139/ssrn.3667878>}.
		Last accessed \today.
		
		
	\end{thebibliography}
	\newpage
	\thispagestyle{empty}
	\clearpage
	
	\section*{Ehrenwörtliche Erklärung}
	
	Ich erkläre hiermit ehrenwörtlich, dass ich die vorliegende Arbeit selbständig angefertigt habe; die aus fremden Quellen direkt oder indirekt übernommenen Gedanken sind als solche kenntlich gemacht.
	\par \bigskip
	\noindent Die Arbeit wurde bisher keiner anderen Prüfungsbehörde vorgelegt und auch noch nicht veröffentlicht.
	
	\vspace{4cm}
	
	\hspace{2cm} Ort, Datum \hfill Unterschrift \hspace{2cm}
	\pagebreak
\end{document}